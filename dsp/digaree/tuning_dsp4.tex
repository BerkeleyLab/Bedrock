% see llrf8/tuning_dsp.tex

\magnification=\magstep1
\input lrdfig.tex
\parindent=0pt
\parskip=6pt
\baselineskip=14.0pt
\font\big=cmbx12 scaled 1200

\centerline{\big LCLS-II LLRF Quench and Detune Revisited}
\smallskip
\centerline{Larry Doolittle, LBNL, 2016-10-25}
\bigskip
Simplify the usual cavity equation of state by including only
one driven port, ignoring $1/Q$ of the probe, and postponing consideration
of beam current.
That leaves us with
$$ {d\vec V\over dt} = a \vec V + b \vec K_1 $$
$$ a = j\omega_d - {1\over 2}\omega_0\left({1\over Q_0}+{1\over Q_1}\right) $$
$$ b = \omega_0\sqrt{(R/Q)\over Q_1} $$
If $Q_0 >> Q_1$, it's possible to write
$$ \Re(a) = -|b|^2 \cdot {1\over 2\omega_0 (R/Q)} $$
where $\omega_0$ and $R/Q$ should be quite well known.

Re-cast the state equation in terms of DSP-processed ADC readings
$\vec M_V$ and $\vec M_K$, unitless where full-scale is represented as 1.
Thus we need calibration constants $c_V$ with units of Volts, and
$c_K$ with units of $\sqrt{\rm Watts}$, so that
$$ \vec V = c_V \cdot \vec M_V $$
$$ \vec K_1 = c_K \cdot \vec M_K $$
Substitute and rearrange the state equation to get
$$ {d\vec M_V\over dt} = a \vec M_V + \left(b{c_K\over c_V}\right) \vec M_K $$
a quick and direct {\it in-situ} experiment (short on-pulse followed by
passive decay, recording waveforms of $M_V$ and $M_K$) can determine
$\Re(a)$ and $\beta \equiv b(c_K/c_V)$ with good accuracy.
This same experiment is
also useful to determine or cross-check the SEL phase offset term.

The relation above gives us
$$ {c_K\over c_V} = {\sqrt{-2\omega_0 (R/Q) \Re(a)}\over |\beta|} $$
which can be cross-checked with what we hope are independent measures
of $c_K$ and $c_V$.

All that is left is to rearrange the state equation one more time to
$$a = {1\over \vec M_V}\cdot \left[{d\vec M_V\over dt} - \beta \vec M_K\right] $$
The FPGA can compute $\vec M_V$ and $\vec M_K$ with good accuracy at
very high rates (up to $\sim$3\thinspace MS/s).  Getting acceptable
accuracy from a meaningful derivative term will take longer;
probably on the scale of 10\thinspace $\mu$s to 100\thinspace $\mu$s.

The strategy for tuning and quench detection comes into focus now.
A DSP engine is fed a continuous stream of $\vec M_V$ and $\vec M_K$
with time separation $T$, from which it can also compute $d\vec M_V/dt$.
It is pre-loaded with a (complex) value of $\beta $.
After each new value of $\vec M_V$ and $\vec M_K$ is loaded, the
equation above is used to compute (complex) $a$.  The imaginary part
of $a$ is the detune frequency $\omega_d$, which can be passed to
the resonance control subsystem.  The real part of $a$ is negative;
if it moves above some pre-loaded threshold (maybe 0.9 of the value of
$a$ found in the setup process), it is interpreted as a dramatic decrease
in $Q_0$, and the quench detector trips.

This mechanism will, at least in theory, work under all modes of
operation: fill, SEL, or closed-loop.  As $|\vec M_V|$ gets smaller,
measurement noise increases, and below some threshold the computation
should be inhibited.  In that case, the tune algorithms should freeze,
and we hope it's safe to assert no-quench.
From a tuning-loop perspective, we want to enforce a narrow range of
voltage validity, and even some dwell-time within that range,
to avoid trying to follow irrelevant Lorentz detuning when the gradient
is not what's needed for beam operations.

%The computations are a bit intricate, but fall well within our ability
%to program in Verilog, maybe best using a microprogrammed engine.
%A good test bench is essential!
Computing the derivative involves a complex difference % 2 adds
followed by multiplying by the scalar representing $1/T$. % 2 mults
One full complex multiply % 4 mults, 2 adds
gives the second term inside the brackets.
Then multiply by the conjugate of $\vec M_V$, % 4 mults, 2 adds
and divide by $|\vec M_V|^2$.  % 2 mults, 1 add, 1 divide, 2 mults
% 1 divide = 4 mult, 2 add, 1 init
All told about 18 fixed-point scalar multiplies and 10 adds or subtracts
are required.

So far this analysis has used s$^{-1}$ as the units for $1/T$, $\beta$,
and therefore $a$.  Some other choice of inverse-time units will
make sense when implementing this with fixed-point DSP.
The software is free to configure that when choosing the numerical values of
$1/T$ and $\beta$ to load into the DSP engine.
With a signed 18-bit representation and a bit-quantum of 0.01\thinspace Hz,
full-scale would be $\pm$1310\thinspace Hz.

Restoring the beam current term to this analysis is not conceptually difficult,
it just requires additional calibration and timing input.  The real un-answered
question is how to measure drifts in the $\beta$ parameter in-place during CW operation.

The restriction that $|\vec V|$ is not small is a troubling one for
a quench detector.  If $\vec V$ is stuck at zero, even with forward power
applied, the cavity could equally well be quenched or far off-resonance.
While a phase-ignoring power balance computation can possibly detect
that condition, it's hard to generalize to the case with beam current,
since the power absorbed is always phase-sensitive.  Let's develop the
power balance technique anyway, and claim that beam current will only
be turned on when $\vec V$ is far from zero, so that the state-vector
approach is running smoothly.
Thus at least one of the two methods will always be valid.

First, refresh our memory that the reverse wave in the waveguide is
$$ \vec R = c \vec V - \vec K $$
(ignoring waveguide losses) where $c = 1/\sqrt{Q_1(R/Q)}$.
This knowledge is helpful when setting up test benches.

The way to account for the cavity's stored energy is through the
equation
$$ U = {V^2\over \omega_0 (R/Q)}$$
$$ {dU\over dt} = 2 \Re\left(\vec V {d\vec V\over dt}\right) \cdot {1\over \omega_0 (R/Q)}$$

The dissipated power in the cavity can then be estimated as
$$ P_{\rm diss} = |\vec K|^2 - |\vec R|^2 - {dU\over dt}$$
If this exceeds some threshold power $P_Q$, we can claim a quench
condition is detected.  This representation purposefully uses the
same notation and input measurements as the state-vector method.

Converting to a hardware perspective, a trip is caused by a positive
sign on the quantity
$$ f_K|\vec M_K|^2 - f_R|\vec M_R|^2 -
  f_V 2\Re\left( \vec M_V \cdot {d\vec M_V\over dt} \right) - f_Q$$
where $f_K = fc_K^2$, $f_R = fc_R^2$, $f_Q = fP_Q$, and
$$ f_V = f {c_V^2 \over \omega_0 (R/Q)}$$
and the free parameter $f$ allows scaling from SI Watts to whatever
internal number system is convenient.  Here we see the software needs
to calibrate and download $f_K$, $f_R$, $f_V$, and $f_Q$.  The DSP
engine needs to perform 9 fixed-point scalar multiplies and
6 adds or subtracts.

The arithmetic described here is more complicated than people usually
associate with direct FPGA programming techniques.  Indeed,
functionality like this has historically been done on a dedicated DSP chip.
Fortunately, FPGAs are big and fast enough now to subsume that programmability
by means of various architectures of soft-cores.  Our use case does not
call even for a general-purpose soft-core, but rather one that can do
a fixed sequence of fixed-point arithmetic (adds, subtracts, multiplies)
every time a new set of RF vector measurements comes in.

Such a streamlined DSP core is on-the-shelf at LBNL; its basic architecture
is shown here.
\medskip
\centerfig{0.75\hsize}{computer}
\smallskip
It is capable of running at higher clock speeds than even the double-rate
DAC clock in the LCLS-II FPGA.  It can perform one batch of this
arithmetic in 80 cycles:  16 to load measurements and parameters,
45 useful arithmetic instructions, 2 output instructions,
and 17 no-op cycles to wait for results to flow through its pipeline.
The actual representation of the jobs to be done (both state-variable
and the power-balance, and an FIR computation of the derivative of cavity
voltage) is written as 17 lines of stylized Python, plus
comments and surrounding overhead (see appendix).  This gets
machine-converted to an 80 $\times$ 22-bit program memory.
As always, a good test bench is essential!

The 80 cycles of computation take little time compared to the many microseconds
needed to develop a decent $dV/dt$ measurement.  The hardware resources
used are tiny both compared to what's available on the chip, and compared
to what would be needed for a direct mapping of the arithmetic to hardware.

\vfill\eject
\baselineskip=13pt
\parskip=0pt
\centerline{\big Appendix: Program Listing}
\medskip

% Cribbed from the TeXbook p. 380-381:
\def\uncatcodespecials{\def\do##1{\catcode`##1=12 }\dospecials}
\def\listing#1{\par\begingroup\setupverbatim\input#1 \endgroup}
\def\setupverbatim{\tt
  \def\par{\leavevmode\endgraf} \catcode`\`=\active
  \obeylines \uncatcodespecials \obeyspaces}
{\obeyspaces\global\let =\ } % let active space = control space
{\catcode`\`=\active \gdef`{\relax\lq}}

\listing{cgen_srf.py}

\bye
