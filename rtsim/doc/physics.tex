\magnification=\magstep1
\parindent=0pt
\baselineskip=13.5pt
\parskip=6pt

%\input lrdfig.tex

The full vector (complex) expression for cavity accelerating voltage $\vec V$,
combining shunt impedance, detuning (primarily microphonics), and beam loading,
is
$$\left(1-j{\omega_d\over \omega_f}\right)\vec V +
 {1\over\omega_f}{d\vec V\over dt} =
 2\vec K_1\sqrt{R_1} - R_c\vec I $$
where $\vec K_1$ is the incident wave amplitude in $\sqrt{\rm Watts}$,
$R_1=Q_1(R/Q)$ is the coupling impedance of port 1,
$\vec I$ is the beam current, $R_c=Q_L(R/Q)$ is the coupling impedance
to the beam, and $\omega_f=\omega_0/2Q_L$ is the mode's bandwidth,
and $\omega_0$ is the nominal resonant frequency of the mode.
$\omega_d=2\pi\Delta f$ is the (time varying) detune frequency,
{\it i.e.}, the difference between actual eigenmode frequency and the accelerator's time base;
%which is proportional to mechanical displacement of the cavity walls;
that term will be discussed more later.
The overall $Q_L$ is given as $1/Q_L=1/Q_0+1/Q_1+1/Q_2$, where $1/Q_0$
represents losses to the cavity walls, $1/Q_1$ represents coupling
to the input coupler, and $1/Q_2$ represents coupling to the field probe.
$(R/Q)$ is the shunt impedance of the mode in Ohms, a pure geometry term
computable for each particular eigenmode using E\&M codes like Superfish.
Physically, shunt impedance relates a mode's stored energy $U$ to the
accelerating voltage it produces, according to
$$U = {V^2 \over (R/Q)\omega_0}~~~.$$

The only assumptions in the above formulation are that the cavity losses
are purely resistive, and thus expressible with a fixed $Q_0$, and that
no power is launched into the cavity from the field probe.  If other
ports have incoming power, there would be additional terms of the same
form as $2\vec K_1\sqrt{R_1}$.

The output wave $\vec E_2$ from the field probe is
$$\vec E_2=\vec V / \sqrt{Q_2(R/Q)}~~~.$$
%The output wave from the fundamental power port combines such a term
%with the prompt reflection of the corresponding input wave,
%$$\vec E_1=\vec V / \sqrt{Q_1(R/Q)} - \vec K_1~~~.$$
The discussion so far applies independently to every cavity eigenmode.
Each such mode has its own value of $\vec V$, $\omega_d$, $(R/Q)$, $Q_i$,
and therefore $\omega_f$ and $R_C$.  The fields from all the eigenmodes
superimpose.  If one assigns the subscript $\mu$
to a particular such mode, the expression for emitted (a.k.a.~reflected)
wave travelling outward from the fundamental port includes a prompt reflection
term, yielding
$$\vec E_1=\sum_\mu \vec V_\mu / \sqrt{Q_{\mu 1}(R/Q)_\mu} - \vec K_1~~~.$$

It's possible to rewrite equation (1) in the frame of the eigenmode itself.
Define $\vec S$ such that $\vec V = \vec S e^{j\theta}$,
where $d\theta/dt = \omega_d$, then
%$${d\vec V\over dt} = {d\vec S\over dt}e^{j\theta} +
%  \vec S \cdot j\omega_d e^{j\theta}$$
$$\left(1-j{\omega_d\over \omega_f}\right)\vec S e^{j\theta} +
 {1\over\omega_f}\left({d\vec S\over dt}e^{j\theta} +
  \vec S \cdot j\omega_d e^{j\theta}\right) =
 2\vec K_1\sqrt{R_c} - R_c\vec I $$
$${d\vec S\over dt} = -\omega_f\vec S  + \omega_f e^{-j\theta}\left(
 2\vec K_1\sqrt{R_c} - R_c\vec I \right)$$
This state-variable equation is a pure low-pass filter, an advantage especially
in the FPGA implementation.

%\vfill\eject
These electromagnetic fields interact mechanically.
Each mode's fields generate a force proportional to $V_\mu^2 = |\vec V_\mu|^2$,
and mechanical displacements influence each mode's instantaneous
detune frequency.  Construct the previous section's $\omega_d$ as
a baseline $\omega_{d0}$ from the electrical mode solution
({\it e.g.}, $-2\pi (800\thinspace {\rm kHz})$ for the TTF cavity's
$8\pi/9$ mode), plus a perturbation
$\omega_{\mu}$ contributed from the mechanical mode deflections.
Consider the electrical mode index $\mu$ to include
not only electrical eigenmodes of one cavity, but modes of all cavities in the
mechanical assembly ({\it e.g.}, cryomodule).
Also include the dependence on piezoelectric actuator voltages $V_\kappa$.
Then if the assembly's mechanical eigenmodes are indexed by $\nu$,
mechanical forces $F_\nu$ and displacements $x_\nu$ of those eigenmodes
are related to the electrical system by
$$F_\nu = \sum_\mu A_{\nu\mu} V_\mu^2 + \sum_\kappa B_{\nu\kappa}V_\kappa$$
$$\omega_\mu = \sum_\nu C_{\mu\nu} x_\nu~~~,$$
where $A$, $B$, and $C$ are constant matrices.
These expressions are understood to apply at every time instant;
the quantities $V$, $F$, $x$, and $\omega$ all vary with time.

\hyphenation{re-phrased}
The differential equation governing the dynamics of each mechanical eigenmode
%are given in Laplace form as
is that of a textbook second order low-pass filter.  In Laplace form,
%$${1\over \omega_\nu^2}{d^2 x_\nu\over dt^2} +
%  {1\over Q_\nu \omega_\nu} {d x_\nu\over dt} + x = {F_\nu\over k_\nu}$$
$$k_\nu x_\nu = {F_\nu \over \displaystyle 1 + {1\over Q_\nu}{s\over \omega_\nu} + \left({s\over \omega_\nu}\right)^2}~~~,$$
where $k_\nu$ is the spring constant.
%That second-order Laplace equation for the dynamics of a single
For computational purposes, we want it expressed
%This can be rephrased
in terms of the state-space formulation
$${d\over dt} \left(\matrix{x\cr y}\right) =
  \left(\matrix{a&-b\cr b&a}\right) \left(\matrix{x\cr y}\right) +
  c\cdot\left(\matrix{0\cr F}\right)~~~,$$
where a scaled velocity coordinate $y$ has been introduced.
%but there is nothing physical that depends on it.
Convert the latter equation to Laplace form and solve to get
$$\left(\matrix{x\cr y}\right) =
  -c\left(\matrix{a-s&-b\cr b&a-s}\right)^{-1}\cdot\left(\matrix{0\cr F}\right)~~~.$$
Analytically invert that $2\times 2$ matrix, and multiply out to get
$$x = {-bcF \over (a-s)^2 + b^2}~~~.$$
Equate coefficients with the earlier low-pass filter form,
in the case $Q > {1\over 2}$, to get
$$a\pm jb = \omega\left( {-1\over 2Q} \pm j\sqrt{1-{1\over 4Q^2}}\right) $$
$$c = -{1\over k}\cdot{a^2+b^2 \over b} = - {\omega^2 \over k b}~~~.$$
All the symbols above, including the mechanical resonance frequency $\omega$,
apply to a single mechanical eigenmode, and thus have an implied $\nu$ subscript.

\def\volume{\Phi}
A deeper understanding of the forces and responses of a single
electrical eigenmode $\mu$ of the cavity comes from Slater's perturbation
theory.  For an eigenmode solution $\vec H(\vec r)\sin(\omega_\mu t)$,
$\vec E(\vec r)\cos(\omega_\mu t)$ to
Maxwell's equations in a closed conducting cavity (volume $\volume$),
the stored energy $U$ is given by
$$U = \int_\volume \left[ {\mu_0    \over 4}H^2(\vec r)
                 +  {\varepsilon_0\over 4}E^2(\vec r) \right] d\volume~~~.$$
Suppose a mechanical eigenmode $\nu$ involves small deflections
$x\cdot \vec\xi(\vec r)$, where $x$ gives the amount of deflection, and
the dimensionless quantity $\xi(\vec r)$ represents the mode shape.
Both the force on the mode and the response to a deflection $x$ are
given in terms of the Slater integral
$$F = \int_S \left[ {\mu_0    \over 4}H^2(\vec r)
                 -  {\varepsilon_0\over 4}E^2(\vec r) \right]
   \vec n(\vec r) \cdot \vec\xi(\vec r) dS ~~~,$$
where $\vec n(\vec r)$ is the normal vector to the cavity surface $S$,
and $F$ directly gives the force.
Note in particular the subtraction of $E$ and $H$ terms, contrasted
with the addition in the energy integral.  Also notice the dot product
of the deflection shape with the surface normal.
Then the resonance frequency shift of the electical eigenmode is given by
$$\Delta\omega_\mu = -x\omega_\mu {F\over U}$$
and the force by
$$F = {F\over U} {1\over (R/Q)\omega_\mu}  V^2~~~,$$
where $F/U$ is a property of the electrical eigenmode, independent of amplitude,
with units of~\hbox{m$^{-1}$}.
Thus $A_{\nu\mu} = (F/U)/((R/Q)\omega_\mu)$,
and $C_{\mu\nu} = -\omega_\mu F/U$.

%Looking at a single electrical mode $\mu$,
%The product $C_{\mu\nu} A_{\nu\mu}/ k_\nu$
%gives the low-frequency Lorentz response of $\omega_\mu$ to $V_\mu^2$.
Slater's analysis above lets us express the static Lorentz response of an
electrical mode to its own stored energy as
$${\Delta \omega_\mu\over V^2} = {C_{\mu\nu} A_{\nu\mu}\over k_\nu} = - \left({F\over U}\right)^2 {1 \over k_\nu (R/Q)}$$
correctly showing that this constant is always negative:
the mode's static resonant frequency gets lower as it is filled.
Summing over all mechanical modes $\nu$ gives the total DC response,
often quoted in units of \hbox{Hz/(MV/m)$^2$}.

Using electrical measurements alone,
it's not possible to constrain the scaling of $x_\nu$.
It is therefore helpful to
rescale $x_\nu$ and $F_\nu$ each by a factor of $\sqrt{k_\nu}$,
and eliminate $k_\nu$ from the equations.  Instead of conventional
units (m and N) for $x$ and $F$, they now both have units
of $\sqrt{\rm Joules}$, so that $x\cdot F$ still represents energy.
In this rescaled no-$k$ case,
%${F/U} = \sqrt{-(R/Q)(\Delta \omega/V^2)} $,
$$A_{\nu\mu} = {1\over \omega_0} \sqrt{-{1\over (R/Q)} {\Delta \omega \over V^2}} $$
$$C_{\mu\nu} = -\omega_0 \sqrt{-(R/Q){\Delta \omega \over V^2}}\rlap{~~~.}$$

It is perhaps an unexpected result that the cross-coupling between
cavity modes ({\it e.g.}, excite the $\pi$ mode, measure $\Delta\omega$
for the $8\pi/9$ mode) is quantitatively predicted from measurements of
each mode individually, with the exception of the choice of sign of
the above radicals.  All that is required is confidence that mechanical
modes are correctly identified and non-degenerate.

See also
{\it Ponderomotive Instabilities and Microphonics -- A Tutorial}, J. R. Delayen

\bye
